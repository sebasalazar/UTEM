\documentclass[10pt]{examdesign}
\usepackage{amsmath}
\usepackage{pifont}
% Idioma
\usepackage[spanish]{babel} % Para separar correctamente las palabras de multitud de idiomas
\usepackage[utf8]{inputenc} %Este paquete permite poner acentos directamente y eñes
\usepackage{graphicx}
\SectionFont{\large\sffamily}
\Fullpages
\ContinuousNumbering
%\ShortKey
% Cantidad de Versiones
\NumberOfVersions{1}
\DefineAnswerWrapper{\\ {\bf Respuesta} \\}{}
\SectionPrefix {}
% \IncludeFromFile{foobar.tex}

\class{{\Large Ingeniería de Software}}
\NoKey
\OneKey
\examname{Prueba 1}
\begin{document}

\begin{shortanswer}[title={Pregunta de desarrollo}, rearrange=yes,resetcounter=yes]
{\bf Se pide justificar debidamente cada una de las respuestas, que en total tendrán un valor de 60 puntos.}

% 1
\begin{question}
	Se ha encargado a su empresa desarrollar un proyecto para ``Falaferia''. El proyecto consiste en agregar funcionalidad a una aplicación web que está desarrollado en strut 1 (java 1.5 llegó al final de su vida útil en Octubre de 2009) y oracle 9i, dicha aplicación en un contenedor tomcat 5.
	\newline
	Falaferia ingresó al negocio de la telefonía móvil, como un operador virtual {\bf Falaferia Mobile}, el problema es que actualmente si un cliente desea portarse o comprar un celular a Falaferia, debe ir a una de las pocas tiendas que hay para este propósito, esto se ha traducido en que muy pocas personas han preferido ``Falaferia Mobile''. Después de un complejo proceso de análisis, el área de Marketing ha deducido que si la funcionalidad ``Falaferia Mobile'' se agregara al sistema de retail (que actualmente se ejecuta en todas las tiendas), cualquier tienda de Falaferia podría hacer dichas ventas, y muy probablemente la gente comience a preferirlo más. Para realizar lo mencionado se ha ideado este proyecto.
	\newline
	Ahora bien, la gerencia piensa que el desarrollo de este proyecto es sencillo y ha puesto a un jefe de proyecto muy novato, que recientemente ha comenzado a trabajar en Falaferia ( no tiene experiencia ni en el negocio ni en la problemática), Además, en este proyecto trabajan distintas áreas que colaborarán con partes muy específicas y delicadas a través de webservices, estas áreas son Financiera (parte crediticia), Mobile (la lógicia de telefonía), Marketing (las promociones) y el Core del negocio (que integra todo el retail), todas estas áreas tienen ideas muy distintas sobre la complejidad del proyecto y por lo mismo su nivel de participación es variada. 
	\newline 
	Su empresa ha estudiado el problema y lo clasifica como un proyecto difícil, ya que el proceso tiene muchas aristas, un complejo sistema de perfiles y la interacción con una gran cantidad de sistemas externos.
	\newline
	Su equipo está compuesto por 3 personas, usted que tiene el rol de Jefe de proyecto, un desarrollador muy hábil en web y otro desarrollador con habilidades en base de datos, pero todos son capaces de hacer de todo y se demoran lo mismo en hacer las tareas. La estimación indica que el proyecto tomaría 570 HH. Del total se dedicarán unas 70 HH para la planificación y documentación (Cuyo costo es de 2 UF cada HH), unas 40 HH a Quality Assurance (QA) (1 UF cada HH) y el resto se destinará a Desarrollo (a un costo de 2.5 UF cada HH), cabe destacar que las tareas de Planificación y QA son secuenciales, mientras que el desarrollo está muy bien desglosado y es perfectamente paralelizable. 
	\newline
	El proyecto comenzó mal, nadie respaldó el proyecto y el jefe de proyecto de Falaferia estaba luchando solo, sus contrapartes internas salieron de vacaciones y durante semanas el proyecto avanzaba a ciegas, nadie formalizaba ningún requerimientos y las necesidades eran ambiguas y contradictorias, Usted y su equipo ante la falta de información, trabajaron en base a supuestos. Otra dificultad fue que ningún sistema externo estaba documentado, el conocimiento lo tenían personas muy específicas y sus vacaciones complicaron el avance en la carta gantt. Por otro lado la complejidad del proyecto quedó latente cuando los desarrollos de las distintas áreas involucradas se atrasaron y con el retraso, llegaron las críticas. Situación que se tradujo en la renuncia del jefe de proyecto de Falaferia. Esto motivo que durante 2 semanas el proyecto quedara paralizado. Ante la falta de liderazgo, el área Mobile tomó las riendas del proyecto y obligó a los distintos proveedores (usted incluido) a trabajar en las dependencias de Falaferia Mobile. Para cumplir con los plazos ideales se contrató a un proveedor dedicado a hacer pruebas, el trabajo fue caótico, ya que no existía ningún tipo de orden y la prioridad en los recursos no estaba definida, el código fuente se pasaba de pendrive en pendrive (lo que ocasionaba que más de alguna vez, se perdieran piezas completas o el código se corrompiera) y nadie sabe qué requerimientos están desarrollados y cuáles no. Por lo mismo, el control de avance del proyecto fue fatal, por el afán de cumplir con los plazos se dejó de hacer cualquier tipo de seguimiento y el esfuerzo se centró en desarrollar entregables, que eran desarrollados sin ningún formato, sin ningún orden y muchas veces las herramientas heterogéneas, dificultaban la puesta en producción. 
	\newline
	El proyecto fue un completo fracaso.
	{ \bf
	En su calidad de Ingeniero de Software y en su rol de Jefe de proyecto se pide:
	\begin{itemize}
	 \item El presupuesto del proyecto, asumiendo un margen de ganancias del 20\%. (6 puntos)
	 \item El tiempo ideal de entrega (en días). (12 puntos)
	 \item El tiempo real de entrega (en días). (12 puntos) 
	 \item Determinar las causas del fracaso del proyecto y entregar las medidas correctivas. (30 puntos)
	\end{itemize}
	}

	\begin{answer}
	    {\bf El presupuesto del proyecto}
	    \begin{itemize}
	     \item Planificación: 70 HH * 2 UF/HH = 140 UF (1 punto)
	     \item Desarollo: 460 HH * 2.5 UF/HH = 1150 UF (1 punto)
	     \item QA: 40 HH * 1 UF/HH = 40 UF (1 punto)
	     \item Costo = 1330 UF (1 punto)
	     \item Presupuesto = 1330 UF * 1.2 = 1596 UF (2 puntos)
	    \end{itemize}

	    {\bf Tiempo}
	    \begin{itemize}
	     \item Tiempo secuencial: 70 HH + 40HH = 110
	     \item Tiempo paralelizable: 460 HH / 3 (la cantidad de desarrolladores) = 153.3 se aproxima a 154 HH
	     \item Tiempo lineal: 110 + 154 = 264 HH 
	    \end{itemize}

	    \begin{itemize}
	     \item Tiempo ideal: 264 / 8 = 33 días (12 puntos)
	     \item Tiempo real: 264 / 5 = 52.8 = 53 días. (12 puntos)
	    \end{itemize}

	    {\bf Medidas correctivas}
	    Existen una gran cantidad de factores que motivaron el fracaso del proyecto:
	    \begin{itemize}
	     \item No existe apoyo de alguien importante.
	     \item No se comunicó la importancia del proyecto.
	     \item No se usan metodologías para nada.
	     \item No se mide nada.
	    \end{itemize}
	    (6 puntos)
	    \newline
	    
	    Las medidas de mejora:
	    \begin{itemize}
	     \item Utilizar una metodología para el desarrollo del proyecto (Explicar como usar PMBOK). (4 puntos)
	     \item Formalizar la toma de requerimientos, utilizando un mecanismo que permita medir cuantitativamente el avance (Por ejemplo las hisotiras de usuario). (4 puntos)
	     \item Utilizar una metodología de desarrollo (Explicar como usar Scrum) (4 puntos)
	     \item Utilizar un sistema de control de versiones (Explicar como un sistema de control de versiones facilita el desarrollo). (4 puntos)
	     \item Utilizar herramientas (Explicar las ventajas de usar Redmine, o similar, IDE y herramientas de apoyo para escribir código legible y mantenible) (4 puntos)
	     \item Definir planes de contingencia, ante los distintos riesgos (Identificar Riesgos y explicar los planes de contingencia) (4 puntos)
	    \end{itemize}


	\end{answer}
\end{question}

\end{shortanswer}
\end{document}

