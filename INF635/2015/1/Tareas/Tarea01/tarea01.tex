\documentclass[a4paper,10pt]{article}
\usepackage[utf8]{inputenc}
\usepackage{graphicx} 


%opening
\title{Tarea Computacional 00}
\author{Sebastián Salazar Molina.}
\date{16 de Marzo de 2015}

\begin{document}

\maketitle

\section{El problema del Azar}


Una empresa de juegos de azar ``Al lote, ría'', necesita un software que realice los sorteos de su juego estrella ``Qno''.
\newline
Qno consiste en escoger de una forma completamente aleatorea 14 números de un conjunto finito de números que van del 1 al 25.
\newline
Se le solicita crear un programa que reciba los parámetros a través de la {\bf línea de comandos}, el orden de ejecución es el siguiente:
\begin{itemize} 
 \item Al ejecutar el programa con la opción {\bf -g}. El programa debe generar un archivo separado por punto y coma (;) de extensión csv, que tiene la siguiente estructura: Fecha en formato ISO (yyyy-mm-dd HH:mm:ss) y la lista de numeros (Ejemplo: {\bf 2015-03-16 12:15;01;02;04;06;07;09;11;13;14;16;20;22;23}). En cada ejecución el programa debe agregar la línea del sorteo.
 \item Al ejecutar el programa con la opción {\bf -v} el programa debe mostrar en pantalla la información de los integrantes del grupo y la fecha de compilación, posterior a esto debe terminar.
\end{itemize}


La forma de discriminar las implementaciones, será en función de la probabilidad de aparición de cada número, entre más aleatoreo mejor.


\subsection{Condiciones}

\begin{itemize}
 \item El lenguaje de programación debe ser C o C++. El compilador en que se revisará será gcc (aconsejo usar MinGW en Windows).
 \item El trabajo debe ser realizado por los grupos de 3 personas, este grupo será definitivo a lo largo del semestre.
 \item La fecha de entrega es hasta el 22 de Marzo de 2015 a las 23:55 (hora continental de Chile), deben enviarme un correo electrónico con la dirección de su repositorio GitHub.
 \item Cualquier duda, la pueden expresar en el grupo de Facebook o a mi correo.
\end{itemize}


\end{document}
