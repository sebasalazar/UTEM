\documentclass[a4paper,10pt]{article}
\usepackage[utf8]{inputenc}
\usepackage{graphicx} 


%opening
\title{Proyecto de Ingeniería de Software \newline Juego de Damas}


\author{\author{Académico: Sebastián Salazar Molina. \newline Ayudante: Patricio Pérez Valverde }}
\date{06 de Abril de 2015}

\begin{document}

\maketitle

\section{Damas Chilenas}

\subsection{Objetivos}
El objetivo del juego es dejar sin movimiento al rival, lo cual se puede conseguir por:
\begin{itemize}
 \item Inmovilizar al contrincante, sin que pueda mover ninguna de sus piezas.
 \item Capturar todas las piezas enemigas.
\end{itemize}

\subsection{Tablero y Piezas}

El tablero es de 10x10 casillas colocando en la esquina inferior derecha la casilla de color blanco. Se practica con 15 peones por jugador (blancos y negros respectivamente para cada uno de ellos) situados sobre las casillas negras de las tres primeras filas de cada jugador. El juego se desarrolla por lo tanto sólo sobre las casillas negras. Comienza el juego quien ocupe las piezas blancas.

\subsection{Reglas}

Los peones (a veces llamados ``perros'') avanzan en forma diagonal por el tablero sólo hacia adelante, no pueden comer hacia atrás. Coronan cuando llegan a la linea final, la dama tiene la particularidad de que ``vuela'' (se mueve libremente sobre las diagonales las casillas que se deseen).
\newline
Para capturar una pieza contraria debe estar justo delante o atrás (sólo para la dama) de una de las piezas. Se realiza en diagonal saltando sobre la pieza que queremos {\it comer}, y cayendo en la casilla inmediatamente detrás de ella siguiendo la dirección del movimiento de nuestra pieza. Este movimiento se puede realizar siempre y cuando la casilla final esté libre. Las capturas se pueden encadenar. Esto es, si saltamos una pieza y desde esa posición podemos saltar una y otra, lo hacemos. Es obligación capturar, siendo inválido aquel movimiento que no captura piezas enemigas teniendo posibilidades para hacerlo. En su turno si un jugador dispone de dos o más opciones de captura, deberá optar por aquella que capture mayor número de piezas contrarias.
\newline
La regla de "preferencia dama", es decir, es obligación antes dos o más posibilidades de comida, hacerlo con la dama aunque con el peón puedas comer más número. Si hay dos o más posibilidades de comer y en alguna de esas posibilidades se encuentre una dama, debe hacerlo en el sentido de donde esté la dama, aunque en el otro haya más número de piesas por comer.
\newline
Otra regla es la ``sopladita'' (obligación de captura), si no se cumple el oponente tiene el derecho a su elección de pedir que se coma o retirar del juego la ficha que no comió.



\section{Proyecto}

\begin{itemize}
 \item El lenguaje de programación debe ser C o C++. El compilador en que se revisará será gcc (aconsejo usar MinGW en Windows).
 \item El trabajo debe ser realizado por los grupos de 3 personas definidos.
 \item La fecha de entrega es hasta el 03 de Julio de 2015 a las 23:55 (hora continental de Chile), deben enviarme un correo electrónico con la dirección de su repositorio GitHub.
 \item Cualquier duda, la pueden expresar en el grupo de Facebook o a mi correo.
\end{itemize}


\end{document}
